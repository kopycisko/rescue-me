\documentclass[../main.tex]{subfiles}

\begin{document}
\chapter*{Introduction}
Evolution is a process in nature that is responsible for the adaptation of individuals of a given species to the environment in which they live.
The basis of this process is the survival principle of better adapted individuals and the phenomenon of inheritance and mutation.

Evolutionary algorithms are a family of heuristics that mimic the process of evolution to optimize \cite{davis1991handbook}.
A single element of the solution space is called an individual in them.
Individuals can compare themselves in terms of the value of the optimized function for them, and the relation of minority (for minimization problems) or majority (for maximization problems) represents the relation of being better adapted to the environment.
In addition, mutants and crosses are defined on the individuals to simulate the natural phenomena.
Heuristics consists of multiple processing of the population (ie, collection of individuals) by using each operator with a certain probability.
In each step (referred to in this case as a generation), the results of these operators are included in the current population, and then a new population is selected, used in the next step of the evolutionary algorithm.
To reproduce the principle of survival of the best fit individuals, the next population is chosen with a higher probability of the individuals being better adapted.

In nature, multiplication of many species is closely related to the phenomenon of species division on the sex.
In the available literature on the subject of evolutionary algorithms, it is rare to find works that take into account this aspect of the evolutionary process.
The reason for this is rather the desire to simplify the operation of the heuristics itself, rather than the better quality of the results obtained without taking this aspect (\cite{GGA, SexualGA}).

The purpose of this paper is to develop an evolutionary algorithm that takes into account the sex of individuals and to examine its effectiveness on selected problems and to compare it with the classical evolutionary algorithm and the selected solutions known from the literature.

In addition, the author has developed a formal description of the evolutionary algorithm that takes into account gender.
Based on this, a programming library was created, used to compare the quality of optimization of selected metaheuristics.

\end{document}