\documentclass[../main.tex]{subfiles}

\begin{document}
\chapter{Hardware and enviornmental constraints}
State of the art in underground navigation solutions. Theoretical topic.

*TO be adapted; book-wcin
Requirements stated for communication system for the undeground operations:
* must be intrinsically safe and explosion proof
* should adhere to the ingress protection (IP) standards;
* must be rugged in structure
* must be size flexible
* must have totality in design including cables, power supply unit, base stations, etc
* must be value-added priced;e stations, etc
* must be robust, inexpensive, easy to expand,
and enable fast and secure connections

\section{Phisics related to waves propagation in underground corridor}

* Waves difraction, <tłumienie>
* What are known issues related to wireless communication in underground installations


\section{Uderground navigation - literature preview}
* List of known positioning system in underground installations
* Characteristics of known positioning systems
** Their concept Advantages and disadvantages
** How can be used
** How they perform the communication (physic/hardware aspect)

There are known solutions for location and monitoring people in undergorund installations. They are together named as LAMPS systems - Location and Moinitoring for Personal Safety systems. Those systems use three components in general:
*

\section{Mobile device sensorics - literature preview}
* Why mobile device is good for positioning purposes? What are the factors?
* What sensorics are present in mobile device.
* Which of them will be useful to increase positioning accuracy

\section{Abilities and limitations of mobile device in context of available positioning methods}
* What means of communication (ex. wireless) can be used in context of positioning system
* Battery limitations
* Sensitivity of receivers

\end{document}