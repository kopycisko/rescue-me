\documentclass[../main.tex]{subfiles}

\begin{document}
\chapter{Goals and thesis scope}



\chapter{Introduction}

\section{Underground installation characteristics}

Description of:
* Construction (very briefly):
** how can look like: from complicated (room and phillar) to simple (tunneling)
** distances
* Conditions in therms of light and air.
* What wireless communication methods are available?

Answer questions:
* if we need the navigation in whole installation? if yes, why?
* if we need the navigation only in some places inside installation? if yes, why?
* what factors may require from navigation system its extensive lifetime?

This section covers a short description of underground installations in general that are the environment for the positioning system.

Underground installation therm is a general description of places such as tunnels and shafts that were digged into the earth in purpose of valuable material extraction, transportation, touristics or other reasons. The common phase in those installations is the phase of their creation. There is a need to digg tunnel or shaft at first in order to reach buried ore deposits or just remove not needed rock. Tunnels and shafts are used in this phase to supply material needed to perform exchavation, for personel transportation and rock transportation to the surface. Mining installations are about continous rock exchavation process (creation phase) while the others, like designed for transporation, ends creation phase and moves to the phase of use and maintenance.

What is the common in underground installation is that there are no reference objects like plants, horizon or sun. Corridors and chambers are almost identical, in particular if there is room-and-pillar extraction method used.


As the purpose of underground installation may be different, there are also different environmental characteristics such as diamensions, type of material (rock), amount of dust, how freqent is in use, what means of communication are placed into, what machines (if any) are being used inside.

Dust combined from moisture deposit himself on a substrate, the walls and ceiling covering symbols describing the hallways. It worsens the orientation.

*TO be adapted; book-wcin
Underground mines, which are characterized by their tough working conditions and hazardous environments, require fool-proof mine-wide communication systems. Communication is used to:
* smooth functioning of mine workings
* ensuring better safety
* save the machine breakdown time
* immediate passing of messages from the vicinity of underground working area to the surface for day-to-day normal mining operations
* for speedy rescue operations in case of disaster (ex euicker medical assistance, faster evaluation of the situation in underground)

Mines often rely on several independent communication systems to support different areas of their activity. There are three options for communication signaling:
* Through-the-wire communication (ex. by twisted pair, coaxial, trolley, leaky feeders, and fiber optic cables)
* Through-the-earth communication (very low frequency radio methods)
* Through-the-air communication (wireless communication systems)

In most cases the communication solutions are based on technologies that use wires (lines). Wireless communication techonologies are used in places that are inaccessible (ex. TODO przodki ) or in places affected by disaster where wired communication got broken. It is also havily used for communication purposes with modern underground equipment such as self-propelled mining machines.

***
Nowadays transportation tunnels usually equip thier corridors with wireless communication technologies that is commonly use by people above ground, such as GSM and WiFi.

***
The wireless communication systems used on surface cannot be applied straightaway in underground mines due to high attenuation of radio waves in underground strata. Undergorund radio waves propagation envioronment differ also because of
* presence of inflammable gases,
* hazardous environment,
* complex corridors topology (mines case),
* complex geological structures,





\section{Usage of mobile devices in underground installations}
* Define 'mobile device'

Answer questions:
* If mobile device (smartphone class) can be used in undergorund installations?
* How usage of mobile device in underground installations may differ from usage in normal conditions (outside underground installation)?

\section{Navigation system use-cases}
* Introduce 2-3 ideas with visualizations

CASE# 1 Mobile application to help evacuate the underground installation

Miners do not have maps with themselves during their work. Maps are kept in Sztygarówka (supervisors place).
Current safety regulations does not take new technology into account. Mines do not know where exactly their miners are. Personal safety equipment consists of oxygen masks enabling to survive 50 minutes, and lamps with GLON transmitter, allowing on detection from a few meters.

European Union encourages to search for a good solution for the miners localization, which, in one of the postulates of its set of recommendations for the coal and steel sector ('Personnel Tracking' task). There are solutions for underground localisation but they allows only to approximate miner's position (error can be range from 300 m (range of a single radio receiver) to the distance to the next transmitter).

This work is a response to the lack of a solution to aid the evacuation of a miner from a threatened area. As part of the work, presentations will be made of the positioning of underground systems, and a method of positioning in underground systems will be proposed, supported by measurements made during the test of the solution. Test will focus on stability, repeatability, accuracy and reliability factors. The work will discuss the mining model representation in terms of the location of the reference points, the location of the miner (system user), the safety points and the evacuation exits. The model should allow both the user to navigate to the nearest safety point, taking into account the current state of the corridors, and to allow presentation of the current position in graphical form. As part of the work, a complete model of the solution will be proposed along with the prototype of application for the mobile device. Finally, there will proposed future works that would base on a concept of integration of the location system with the function of remotely updating corridors. There will be provided example use cases.

  Use case studies: fire, smoky corridor
  Use case studies: no electricity (dark, no ventilation)


\end{document}