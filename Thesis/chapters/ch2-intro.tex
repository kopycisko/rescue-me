\documentclass[../main.tex]{subfiles}

\begin{document}
\chapter{Goals and thesis scope}



\chapter{Introduction}

\section{Underground installation characteristics}

Description of:
* Construction (very briefly):
** how can look like: from complicated (room and phillar) to simple (tunneling)
** distances
* Conditions in therms of light and air.
* What wireless communication methods are available?

Answer questions:
* if we need the navigation in whole installation? if yes, why?
* if we need the navigation only in some places inside installation? if yes, why?
* what factors may require from navigation system its extensive lifetime?

\section{Usage of mobile devices in underground installations}
* Define 'mobile device'

Answer questions:
* If mobile device (smartphone class) can be used in undergorund installations?
* How usage of mobile device in underground installations may differ from usage in normal conditions (outside underground installation)?

\section{Navigation system use-cases}
* Introduce 2-3 ideas with visualizations

\end{document}